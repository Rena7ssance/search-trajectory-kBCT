%# -*- coding: utf-8-unix -*-
%%==================================================
%% chapter05.tex for SJTU Master Thesis
%% based on CASthesis

%%==================================================
\chapter{结论}
\label{chap:conclusion}

\section{相似轨迹查询系统设计总结}
\label{sec:conclusion implemention}
本文在相似轨迹查询设计和实现之前,先查阅了相关资料。了解并学习相关与轨迹数据挖掘的基础知识后,开始初步设计出与传统相似轨迹查询不同的相似轨迹处理算法。在问题本质上,我们研究并实现一种通过一组有序或无序的轨迹点从轨迹数据库中查询k条最佳连接这些轨迹点的轨迹,从地理语义上而言,我们可以将这k条最佳连接轨迹等价看我们目标查找的k条最相似轨迹。这类相似轨迹查询不同于之前已有的工作,它能在轨迹推荐、交通路况分析和生物站点分配等等中有着广泛的应用空间。在实际情景和本文上下文中,我们都限定这一组查询点集的数目相对较少,这一限定条件使得我们可以利用一些空间索引和查询方法来实现我们的搜索过程。轨迹简化这一预处理技术在本文实现过程中起着很重要作用,我们可以即利用轨迹简化技术来减少我们对设备存储空间的需求,也可以在查询相似轨迹的过程中将一条轨迹做为输入,后者便于用户用轨迹输入取代繁琐的查询点集。

本文设计出增长型k最近邻相似轨迹查询方法。在定义适合的轨迹相似度方程之后,主体算法基于k最近邻分类查询算法,结合备选和筛选的处理思路来进行查询。在k最近邻查询中,使用便于空间搜索的R树数据结构来完成备选过程处理;定义轨迹相似度的上界与下界,在循环过程中以上下界为剪枝条件来不断筛选出符合条件的k条相似轨迹。在优化过程中,根据地理位置点语义上重要性的不同,因地制宜动态设定查询范围,以更快速地完成查询过程。在初步了解分布式集群处理理念后,本文将相似轨迹查询应用与分布式处理框架相结合,可以将输入范围扩大至整条轨迹的同时以保证搜索数据,借助\emph{MapReduce}的编程处理框架和已有的分布式存储技术,本文也保证了更准确的查询结果。

我们通过现实生活中采集的GPS轨迹数据作为数据集,从实际应用需求的角度设计出符合用户需求和数据特点的相似轨迹查询系统,依照对系统本身的需求对系统功能、界面和性能模块做针对性的设计和规划,并对实现好的相似轨迹查询系统做性能评价。

\section{未来工作展望}
\label{sec:conclusion future}
轨迹数据挖掘的系统框架在绪论章节有大致的讨论,在本文所设计的相似轨迹查询算法设计中,仅仅通过轨迹简化预处理和轨迹查询完成了初步任务。在实际应用中,对轨迹数据点的修正、降噪预处理、选择更合理的轨迹索引结构以及在相似轨迹查询中辅以轨迹模式挖掘与轨迹分类都能够在一定程度上对轨迹准确性或是算法性能有提高。

其次对于分布式框架而言,本文所设计的分布式操作仅借助简单的操作,以完成初步的查询任务。在时间允许的情况下可以从\emph{Spark}提供的丰富的弹性分布式数据集操作接口去更合理的安排数据处理流程和方式。相对于本文通过单一设备搭建本地集群环境而言,将工作环境移植备性能更高的集群环境并且按照节点数目和数据大小设定自定义集群处理环境也能够大幅度提高。

