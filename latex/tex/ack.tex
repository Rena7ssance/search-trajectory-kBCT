%# -*- coding: utf-8-unix -*-
\begin{thanks}
本科四年级下学期几个月的时间内,我从毕业设计的论题提交到现在最终完成这一份毕业论文,从一开始对轨迹数据挖掘领域毫无涉及不知道自己该从什么角度下手,到现在能够大概了解轨迹数据挖掘的系统框架并实践自己在相似轨迹数据查询这一领域的初步应用,努力所带来的收获远超过这一篇毕业论文所提及的内容。在此期间,十分感觉在我奋斗过程中不断鼓励我指导我的导师,为我提供各种帮助和建议的同学。

感谢我毕业设计的导师朱燕民老师,在我完成自己毕业设计开始会为我提供入手的方向和,不厌其烦地在我陷入不知所措的时候提供最及时的指导。朱燕民老师每周对毕业设计进度的关心让我在完成毕业设计的过程中脚踏实地,一步一步完成自己的既定目标,并在每周的汇报过程中基于我当前阶段最中肯的建议,让我在这一过程中时时刻刻都充满着前进的动力。

感谢徐亚楠学长、张博文学长和黄凯欣学长在我实现相似轨迹查询和学习分布式计算过程中给予的帮助和指导,让我能够从正确的方向了解并学习相关领域的知识。感谢朱灏和吴思禹两位同学在实现毕业设计中提供的帮助与建议。

最后感谢上海交通大学对我本科四年的培养,让我从四年的本科教育学习中逐渐入门计算机这一广大的知识领域,明晰了自己未来发展的方向,让我在有限的时间里为未来的求学生涯与工作领域打下了扎实的知识基础。



\end{thanks}
